\documentclass[12pt]{article}

\usepackage{enumerate}
\usepackage{mathtools}
\usepackage{listings}
\usepackage[top=5em, bottom=5em, left=5em, right=5em]{geometry}

\DeclarePairedDelimiter\floor{\lfloor}{\rfloor}

\title {Assignment 1}

\author {
	Hendrik Werner s4549775
	\and Constantin Blach s4329872
}

\begin{document}
\maketitle

\section{} %1
\subsection{} %1.1
\begin{enumerate}[a]
	\item %a
	The algorithm can handle $\floor{\sqrt{1000000} * 60} = 60000$ elements per minute.

	\item %b
	\item %c
	The algorithm can handle $\floor{\frac{\log{1000000}}{\log{2}} * 60} = 1195$ elements per minute.

	\item %d
	The algorithm can handle $\floor{10000 * 60 = 600000}$ elements per minute.

	\item %e
	The algorithm can handle $\floor{\sqrt[100]{1000000} * 60} = 68$ elements per minute.

	\item %f
	The algorithm can handle $\floor{\frac{\log{1000000}}{\log{4}} * 60} = 597$ elements per minute.

	\item %g
	The algorithm can handle $\floor{10^6 * 60} = 60000000$ elements per minute.
\end{enumerate}

\subsection{} %1.2
\begin{tabular}{|l|r|}
	\hline
	Question & T(rue) / F(alse) \\
	\hline
	a & T\\
	b & T\\
	c & T\\
	d & F\\
	e & T\\
	f & T\\
	g & T\\
	h & F\\
	\hline
\end{tabular}

\subsection{} %1.3
\begin{enumerate}[a]
	\item %a
	\begin{enumerate}[a]
		\item % a (2a)
		We want to prove $f \in O(g)$ for $g = n$. This means proving that $f(n) \leq c * g(n)$ for all $n > n_0$ for $c > 0$ for $f(n) = n + 1$.

		We choose $c = 2, n_0 = 1$ and get $n + 1 \leq 2n$ for all $n \geq 1$.

		\item % b (2b)
		We want to prove $f \in O(g)$ for $g = n$. This means proving that $f(n) \leq c * g(n)$ for all $n > n_0$ for $c > 0$ for $f(n) = 2n$.

		We choose $c = 2, n_0 = 0$ and get $2n \leq 2n$ for all $n \geq 0$.
	\end{enumerate}

	\item %b
\end{enumerate}

\subsection{} %1.4
The constant is 4 operations: $c = 4$

\begin{enumerate}[a]
	\item %a
	In the worst case scenario v does not contain x. The whole array has to be iterated over.

	Each loop where x is not found takes 4 operations: $l = 4$

	In the worst case scenario the function will take $T(n) = c + ln = 4 + 4n$ operations.

	\item %b
	In the best case scenario $n = 0$ so the loop is never executed as the first check fails. This means it only takes 1 operation.

	The function will take $c + 1 = 4 + 1 = 5$ operations. This does is not a function of $n$ because in the best case $n = 0$.
\end{enumerate}

\subsection{} %1.5
\begin{enumerate}[a]
	\item %a
	The least $O$-class is $O(n^2)$ because $n$ is decremented on each iteration meaning it takes $n$ steps and each step costs $3n$ operations: $O(n * 3n) = O(n * n) = O(n^2)$

	\item %b
	The least $O$-class is $O(\log n)$ because $n$ gets halfed on each iteration meaning it takes $\log n$ steps and each step costs a constant amount of operations: $O(\log n * 1) = O(\log n)$

	\item %c
	The least $O$-class is $O(\log n)$ because $n$ is reduced to a third on each iteration meaning it takes $\log n$ steps and each step costs a contant amount of operations: $O(\log n * 1) = O(\log n)$

	\item %d
	The least $O$-class if $O(n \log n)$ because $n$ is reduced to a quarter on each iteration meaning that it takes $\log n$ steps and each step costs $n$ operations: $O(\log n * n) = O(n \log n)$
\end{enumerate}

\subsection{} %1.6
\lstinputlisting{code/findKthSmallest.groovy}

\subsection{} %1.7
\lstinputlisting{code/findKeyEqualsValue.groovy}

\end{document}