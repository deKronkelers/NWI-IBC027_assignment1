\documentclass[12pt]{article}

\usepackage{enumerate}
\usepackage{mathtools}
\usepackage{listings}
\usepackage[top=5em, bottom=5em, left=5em, right=5em]{geometry}

\DeclarePairedDelimiter\floor{\lfloor}{\rfloor}

\title {Assignment 1}

\author {
	Hendrik Werner s4549775
	\and Constantin Blach s4329872
}

\begin{document}
\maketitle

\section{} %1
\subsection{} %1.1
\begin{enumerate}[a]
	\item %a
	\item %b
	\item %c
	The algorithm can handle $\floor{\frac{\log{1000000}}{\log{2}} * 60} = 1195$ elements per minute.

	\item %d
	The algorithm can handle $10000 * 60 = 600000$ elements per minute.

	\item %e
	\item %f
	The algorithm can handle $\floor{\frac{\log{1000000}}{\log{4}} * 60} = 597$ elements per minute.

	\item %g
	The algorithm can handle $10^6 * 60 = 60000000$ elements per minute.
\end{enumerate}

\subsection{} %1.2
\begin{tabular}{|l|r|}
	\hline
	Question & T(rue) / F(alse) \\
	\hline
	a & T\\
	b & T\\
	c & T\\
	d & \\
	e & T\\
	f & T\\
	g & T\\
	h & F\\
	\hline
\end{tabular}

\subsection{} %1.3
\begin{enumerate}[a]
	\item %a
	\item %b
\end{enumerate}

\subsection{} %1.4
The constant is 4 operations: $c = 4$

\begin{enumerate}[a]
	\item %a
	In the worst case scenario v does not contain x. The whole array has to be iterated over.

	Each loop takes a maximum of 5 operations: $l = 5$

	In the worst case scenario the function will take $c + ln = 4 + 5n$ operations.

	\item %b
\end{enumerate}

\subsection{} %1.5

\subsection{} %1.6
\lstinputlisting{code/findKthSmallest.groovy}

\subsection{} %1.7
\lstinputlisting{code/findKeyEqualsValue.groovy}

\end{document}